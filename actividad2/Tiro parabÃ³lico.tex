\documentclass{article}

% set font encoding for PDFLaTeX or XeLaTeX
\usepackage{ifxetex}
\ifxetex
  \usepackage{fontspec}
\else
  \usepackage[T1]{fontenc}
  \usepackage[utf8]{inputenc}
  \usepackage{lmodern}
\fi

% used in maketitle
\title{Tiro parabólico}
\author{David Hernández\\ Departamento de Física \\ Universidad De Sonora}

% Enable SageTeX to run SageMath code right inside this LaTeX file.
% documentation: http://mirrors.ctan.org/macros/latex/contrib/sagetex/sagetexpackage.pdf
% \usepackage{sagetex}

\begin{document}
\maketitle
\section {Tiro Parabólico}

Se denomina movimiento parabólico, al movimiento realizado por cualquier objeto cuya trayectoria describe una parábola. Se corresponde con la trayectoria ideal de un proyectil que se mueve en un medio que no ofrece resistencia al avance y que está sujeto a un campo gravitatorio uniforme. El movimiento parabólico es un ejemplo de un movimiento realizado por un objeto en dos dimensiones o sobre un plano. Puede considerarse como la combinación de dos movimientos que son un movimiento horizontal uniforme y un movimiento vertical acelerado.


El tiro parabólico tiene las siguientes características:


-Conociendo la velocidad de salida (inicial)

-el ángulo de inclinación inicial y la diferencia de alturas (entre salida y llegada) se conocerá toda la trayectoria.

-Los ángulos de salida y llegada son iguales (siempre que la altura de salida y de llegada sean iguales)

-La mayor distancia cubierta o alcance se logra con ángulos de salida de 45º.

-Para lograr la mayor distancia fijado el ángulo el factor más importante es la velocidad.

-Se puede analizar el movimiento en vertical independientemente del horizontal.

\subsection{Tiempo de vuelo}
Es posible conocer el tiempo de vuelo del cuerpo cuando se nos da la rapidez inicial y el ángulo inicial


Los datos obtenidos en la Tabla uno se obtuvieron con el codigo posteriormente expuesto.

\begin{verbatim}




program projectile
  implicit none

  ! definimos constantes
  real, parameter :: g = 9.8
  real, parameter :: pi = 3.1415927

  ! definimos las variables
  real :: a
  real :: t, vo
  write(*,*) "calcularemos el tiempo de vuelo de un proyectil  con un ángulo
  y velocidad inicial"

  write(*,*) "dame la velocidad inicial y  el ángulo"
  read(*,*) vo, a

  !convirtiendo el ángulo a radianes
   a=a*pi/180.0

  !para calcular el tiempo utilizaremos
  !t=(2vo*sin(a))/g

  t=((2*vo*sin(a))/g)

  write(*,*) "t: ", t


  endprogram projectile
  \end{verbatim}


\begin{table}[]
\centering
\caption{ Tiempo de vuelo}
\label{my-label}
\begin{tabular}{|l|l|l|l|}
\hline
\textbf{} & \textbf{Ángulo} & \textbf{Vo} & \textbf{Tiempo} \\ \hline
\textbf{} & \textbf{45}     & \textbf{32} & \textbf{4.61}   \\ \hline
\textbf{} & \textbf{32}     & \textbf{8}  & \textbf{0.61}   \\ \hline
\textbf{} & \textbf{45}     & \textbf{0}  & \textbf{0.00}   \\ \hline
\end{tabular}
\end{table}



\subsection{Altura Máxima}

Es La altura máxima de un del objeto se puede encuentra cuando sabemos que en ese instante el objeto se detiene, sabiendo que la V=0 en ese punto.

Los datos obtenidos en la tabla 2 se lograron obtener mediante el código que se muestra acontinuación, cabe resaltar que la resistencia del aire fue despreciable y además los datos fueron otorgados por un usuario.

\begin{verbatim}
 
 
 program altura_maxima
   Implicit none
    !declaramos constantes
     real, parameter::pi=3.1415927
     real, parameter::g=9.8
      !declararemos variables
       real::a 
       real::h , vo
        write(*,*) "calcularemos la altura maxima"
	write(*,*) "no tomaremos en cuenta la resistencia del aire"
	write(*,*) "Dame la velocidad inicial y un ángulo"
	read(*,*) vo , a
	 !convirtiendo el ángulo a radianes
	 a=(a*pi)/180
	 !para calcular la altura utilizaremos
	 !h=((vo**2)sin(a)/2g)
	 h=(vo*vo)*(sin(a)*sin(a))/2*g
	  write(*,*) "h:", h
	  endprogram altura_maxima

\end{verbatim}

\begin{table}[]
\centering
\caption{Altura Máxima}
\label{my-label}
\begin{tabular}{|l|l|l|l|}
\hline
\textbf{} & \textbf{Ángulo} & \textbf{Vo} & \textbf{Altura} \\ \hline
\textbf{} & \textbf{45}     & \textbf{30} & \textbf{2205}   \\ \hline
\textbf{} & \textbf{12}     & \textbf{5}  & \textbf{5.35}   \\ \hline
\textbf{} & \textbf{25}     & \textbf{1}  & \textbf{0.87}   \\ \hline
\end{tabular}
\end{table}

\subsection{Distancia recorrida }
Como sabemos el tiro parabólico tiene dos componentes, una en el eje x y otra en el eje Y, en este caso nos interesa saber la distancia recorrida por el objeto en el eje x.   la tabla 3 se logró obtener mediante el código posteriormente expresado, donde el usuario nos otorgaba la Vo ya que sabemos que debemos manejar el ángulo  en este caso como 45 grados para que nos sea expresada la distancia máxima recorrida.

\begin{verbatim}







program x_max
 implicit none
 !declaramos constantes
 real, parameter:: pi=3.1415927
 real, parameter:: g=9.8
 real, parameter:: u=45
 !declaramos variables
 real::vo,d,a
 !utilizaremos el ángulo como 45°
 write(*,*) "calcularemos distancia maxima (x) de un objeto"

 a=u*pi/180
 !la fórmula para calcular la distancia máxima es d=((vo*vo)/g)*sin2u
 write(*,*) "dame el valor de la velocidad inicial"
 read(*,*) vo
 d=((vo*vo)/g)*sin(2*a)
 write(*,*) "d: ",d,""
 endprogram x_max
\end{verbatim}

\begin{table}[]
\centering
\caption{ distancia recorrida}
\label{my-label}
\begin{tabular}{|l|l|l|l|}
\hline
\textbf{} & \textbf{Ángulo} & \textbf{Vo} & \textbf{Distancia (x)} \\ \hline
\textbf{} & \textbf{45}     & \textbf{23} & \textbf{53.97}         \\ \hline
\textbf{} & \textbf{45}     & \textbf{5}  & \textbf{2.55}          \\ \hline
\textbf{} & \textbf{45}     & \textbf{7}  & \textbf{5.00}          \\ \hline
\end{tabular}
\end{table}

\section{Bibliografía}
\begin{verbatim}
https://en.wikipedia.org/wiki/Projectile_motion
https://es.wikipedia.org/wiki/Movimiento_parab%C3%B3lico
\end{verbatim}


\end{document}
