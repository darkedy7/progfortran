\documentclass{article}

% set font encoding for PDFLaTeX or XeLaTeX
\usepackage{ifxetex}
\ifxetex
  \usepackage{fontspec}
\else
  \usepackage[T1]{fontenc}
  \usepackage[utf8]{inputenc}
  \usepackage{lmodern}
  \usepackage{graphicx}
  \usepackage{float}

\fi

% used in maketitle
\title{Evaluación1}
\author{Eduardo Hndz\\Lic. En Física\\Universidad De Sonora\\ Progración y Lenguaje Fortran}

% Enable SageTeX to run SageMath code right inside this LaTeX file.
% documentation: http://mirrors.ctan.org/macros/latex/contrib/sagetex/sagetexpackage.pdf
% \usepackage{sagetex}

\begin{document}
\maketitle
\section{Sphere}
Se nos otorgó un código el cual servía para calcular el área de un cilindro, del cual se pedía, modificarlo para posteriormente calcular el área y volumen de una esfera.
Tuve que probar el código con distintos datos para ver cómo funciona y al momento de querer calcular el V y el A de la esfera tuve problemas con el arreglo acerca del número de dígitos que podría aceptar el valor del área obtenido. El código quedó de la siguiente manera:
\begin{verbatim}
program sphere

! Calculate the surface area of a cylinder.
!
! Declare variables and constants.
! constants=pi
! variables=radius squared 

  implicit none    ! Require all variables to be explicitly declared

  integer :: ierr
  character(1) :: yn
  real :: radius, area , volume
  real, parameter :: pi = 3.141592653589793

  interactive_loop: do

!   Prompt the user for radius and height
!   and read them.

    write (*,*) 'Enter radius.'
    read (*,*,iostat=ierr) radius

!   If radius and height could not be read from input,
!   then cycle through the loop.

    if (ierr /= 0) then
      write(*,*) 'Error, invalid input.'
      cycle interactive_loop
    end if

!   Compute area.  The ** means "raise to a power."

    area = 4*pi*radius*radius

    volume=(4./3.)*(pi*radius*radius*radius)

!   Write the input variables (radius, height)
!   and output (area) to the screen.

    write (*,'(1x,a7,f20.6,5x,a7,f20.6,5x,a7,f20.6)') &
      'radius=',radius,'area=',area, 'volume=', volume

    yn = ' '
    yn_loop: do
      write(*,*) 'Perform another calculation? y[n]'
      read(*,'(a1)') yn
      if (yn=='y' .or. yn=='Y') exit yn_loop
      if (yn=='n' .or. yn=='N' .or. yn==' ') exit interactive_loop
    end do yn_loop

  end do interactive_loop

end program sphere

\end{verbatim}
\section{Medias y Sumatorias}
Aquí se nos pidió calcular la Media Harmónica, la Media Aritmética y la sumatoria de numeros dados por el usuario otorgador un código en el cual el problema que tuve fue que combinaba distintas variables, reales y enteras juntas, lo cual crasheaba mi programa.
el código quedó como sigue:
\begin{verbatim}
program summation
implicit none
integer ::  a
real:: sum, n,am,hm,m
print*, "This program performs summations. Enter 0 to stop."
open(unit=10, file="SumData.DAT")
!dividir enteros entre enteros
!nunca combinar números
sum = 0.0
n=0.0
m=0.0
do 
 print*, "Add:"
 read*, a
 if (a == 0) then
  exit
else
    sum = sum + a
    n=(n+1)
   m=m+1.0/a
end if
   am=sum/n
   hm=n/m
 write(10,*) a
end do
   

   print*, "Summation =", sum
   print*, "Arithmetic mean =", am
   print*,"Harmonic mean=" ,hm
write(10,*) "Summation =", sum
write(10,*) "Arithmetic mean=", am
write(10,*) "Harmonic mean=", am
   
close(10)
end!
\end{verbatim}
Los datos obtenidos se exportaban a un documento de texto el cual quedó de la siguiente manera:
\begin{verbatim}
           2
           3
           4
           5
           6
           7
           8
           9
          10
 Summation =   55.0000000    
 Arithmetic mean=   5.50000000    
 Harmonic mean=   5.50000000    
\end{verbatim}
\end{document}
