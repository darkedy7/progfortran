\documentclass{article}

% set font encoding for PDFLaTeX or XeLaTeX
\usepackage{ifxetex}
\ifxetex
  \usepackage{fontspec}
\else
  \usepackage[T1]{fontenc}
  \usepackage[utf8]{inputenc}
  \usepackage{lmodern}
\fi

% used in maketitle
\title{Actividad 1}
\author{Eduardo Hernández \\ Departamento de Física \\ Universidad de Sonora}
\date { 31 de agosto de 2017}

% Enable SageTeX to run SageMath code right inside this LaTeX file.
% documentation: http://mirrors.ctan.org/macros/latex/contrib/sagetex/sagetexpackage.pdf
% \usepackage{sagetex}

\begin{document}
\maketitle
% son comentarios
\section {Intruducción a curso de fortran}

\subsection{Comandos de Bash}
\subsubsection{descripción de comandos}

%pegamos lo copiado que se realizó anteriormente en emacs
\begin{verbatim}
Descripción comandos
      ls : enlista las carpestas y archivos en una dirección específica o
      por default. así mismo tiene sus variables que te permiten ver no
      solo archivos o carpetas, si no también documentos ocultos.
 
          Ej.- ls -al

     cp: sirve para copiar ficheros y/o archivos. tiene diferentes
     variables mediante las cuales puedes copiar solo la extensión del
     archivo, copiar sin reemplazar,etc.

         Ej: cp notas.txt  -n Música


     pwd: nos indica la ubicación/directorio  donde nos encotramos
     actualmente.
     ej.
     user@bash: pwd
     porgfortran/notas


    cd: nos ayuda a cambiar de directorio, ya sea de forma específica
    o no, podemos utilizar tildé,punto ó doble punto.
    ej.
    /progfortran/notas$
    cd ..
    /profortran


    file: te indica que tipo de documento es.
    ej.
    /fortran/notas ls
    notas.txt actividad1.txt
    tipe actividad1.txt
    utf-8 unicode text


    Cat: concatena (copia e imprime lo deun documento o un texto en la
    terminal)

    ej.
    /fortran/notas ls
    notas.txt
    cat notas.txt
    Actividad uno
    realizar una lista de comandos en emacs


    Man: nos da una descripción específica de algún comando en
    particular, así como sus variantes,descrición, modo de uso, etc.

     Ej.
       cp - copy files and directories

      SYNOPSIS
       cp [OPTION]... [-T] SOURCE DEST
       cp [OPTION]... SOURCE... DIRECTORY
       cp [OPTION]... -t DIRECTORY SOURCE...

      DESCRIPTION
       Copy SOURCE to DEST, or multiple SOURCE(s) to DIRECTORY.

       Mandatory arguments to long options are mandatory for short options too.

       -a, --archive
              same as -dR --preserve=all

       --attributes-only
              don't copy the file data, just the attributes


    mkdir: crea un directorio y/o suple uno.

       ej.-
          /admin/edyhndz
          mkdir Carros
          /admin/edyhndz ls
          carros notas.txt


   rmdir: Elimina un directorio.


     ej.-
          /admin/edyhndz
          /admin/edyhndz ls
          carros notas.txt
	  /admin/edyhndz rmdir carros
	  notas.txt



\end{verbatim}

\subsubsection{Ecuación}

%utilizar comando en par de ecuaciones

\begin{equation}
F=ma=m\frac{dv}{dt}
\end{equation}

\begin{equation}
W=\int_{a}^{b}Fds
\end{equation}


\section{Uso de latex }

El aprendizaje de la herramienta latex es muy práctico  ya que la redacción de papers científicos se facilita mediante de la introducción de equaciones, datos,etc.


\end{document}
